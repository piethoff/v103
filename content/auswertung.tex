\section{Auswertung}
\label{sec:Auswertung}
%
\subsection{Bestimmung des Elastizitätsmoduls eines runden Messingtabs bei einseitiger Belastung}
Der Stab wird dabei so eingespannt, dass dieser $\SI{50}{\centi\meter}$ überhängt.
An diesen wird ein Gewicht von $m=\SI{742.1}{\gram}$ angehangen.
Auf den Stab wirkt also eine Kraft von:
\begin{equation}
  F= m g = \SI{7.28}{\newton}
\end{equation}
Die Messung wird wie in Kapitel 2 beschrieben durchgeführt und die so erhaltenen Werte sind in 
Tabelle \ref{tab:r_einseit} aufgetragen.
\begin{table}[H]
    \centering
    \caption{Messwerte des runden Stabs bei einseitiger Belastung.}
    \label{tab:r_einseit}
    \begin{tabular}{S[table-format=2.0] S[table-format=2.2] S[table-format=2.2] }
        \toprule
        {$x/\si{\centi\meter}$} & {$D_0/\si{\milli\meter}$} & {$D_m/\si{\milli\meter}$} \\
        \midrule
        4     & 11.06   & 11.02    \\
        6     & 11.11   & 11.07    \\
        8     & 11.20   & 11.10    \\
        10    & 11.25   & 11.07    \\
        12    & 11.30   & 11.03    \\
        14    & 11.31   & 10.96    \\
        16    & 11.34   & 10.86    \\
        18    & 11.38   & 10.70    \\
        20    & 11.39   & 10.53    \\
        22    & 11.42   & 10.34   \\
        24    & 11.59   & 10.33    \\
        26    & 11.58   & 10.13    \\
        28    & 11.55   &  9.81    \\
        30    & 11.52   &  9.53    \\
        32    & 11.51   &  9.21    \\
        34    & 11.43   &  8.96    \\
        36    & 11.41   &  8.59   \\
        38    & 11.38   &  8.23 \\
        40    & 11.41   &  7.80 \\
        42    & 11.32   &  7.46 \\
        44    & 11.35   &  6.85 \\
        46    & 11.25   &  6.68 \\
        48    & 11.20   &  6.01 \\
        \bottomrule
    \end{tabular}
\end{table}
\noindent Die Werte für $D(x)$ und $g(x)=Lx^2-\frac{x^3}{3}$ sind in Abbildung \ref{fig:plot1} 
aufgetragen und durch diese wurde eine Ausgleichsgerade mittels linearer Regression gezogen.
\begin{figure}
    \centering
    \includegraphics{build/r_einseit.pdf}
    \caption{Messwerte und Ausgleichsgerade für die einseitige Belastung des ruden Stabs.}
    \label{fig:plot1}
\end{figure}
%
\noindent Die Ausgleichsgerade besitzt eine Steigung von $m=\SI[per-mode=reciprocal]{0.0663\pm 
0.002}{\per\meter\squared}$.
Der Elastizitätsmodul lässt sich nach Gleichung \eqref{eq:e_aus} durch
\begin{equation}
  E=\frac{F}{2mI}
\end{equation}
bestimmen.
$I$ ist hier das Flächenträgheitsmoment, welches sich wie folgt berechnen lässt.
Aufgrund der Symetrie werden Zylinderkoordinaten verwendet:
\begin{equation}
  I=\int_0^{2\pi} \int_0^R r^3 \sin^2{\phi} dr d\phi=\frac{\pi}{4}R^4
\end{equation}
Der Stab besitzt einen Durchmesser von $Q=\si[per-mode=reciprocal]{\per\centi\meter}$, wodurch 
sich ein Flächenträgheitsmoment ergibt von:
\begin{equation*}
  I=\SI{4.91e-10}{\meter\tothe{4}}
\end{equation*}
Mit den so errechneten Werten ergibt sich ein Elastizitätsmodul von:
\begin{equation*}
  E=\SI{1.118\pm0.003e11}{\newton\per\meter\squared}
\end{equation*}
Die Abweichung errechnet sich hier über Gaußsche Fehlerfortpflanzung.
%
\subsection{Bestimmung des Elastizitätsmoduls eines eckigen Aluminiumstabs bei einseitiger Belastung}
Die Messung gestaltet sich analog zu der vorausgegangenen.
An diesen wird die Auslenkung über ein Gewicht mit der Masse $m=\SI{1211.5}{\gram}$ realisiert.
Auf den Stab wirkt dann eine Kraft von $F=\SI{11.88}{\newton}$.
Die erhaltenen Messwerte aus Tabelle \ref{tab:e_einseit} sind in Abbildung \ref{fig:plot2} aufgetragen.
\begin{table}[H]
    \centering
    \caption{Messwerte des eckigen Stabs bei einseitiger Belastung.}
    \label{tab:e_einseit}
    \begin{tabular}{S[table-format=2.0] S[table-format=2.2] S[table-format=2.2] }
        \toprule
        {$x/\si{\centi\meter}$} & {$D_0/\si{\milli\meter}$} & {$D_m/\si{\milli\meter}$} \\
        \midrule
        4     & 10.56   & 10.45    \\
        6     & 10.76   & 10.54    \\
        8     & 10.93   & 10.57    \\
        10    & 11.08   & 10.55    \\
        12    & 11.17   & 10.50    \\
        14    & 11.23   & 10.43    \\
        16    & 11.31   & 10.35    \\
        18    & 11.36   & 10.19    \\
        20    & 11.43   & 10.04    \\
        22    & 11.51   &  9.86   \\
        24    & 11.52   &  9.65    \\
        26    & 11.54   &  9.39    \\
        28    & 11.56   &  9.10    \\
        30    & 11.70   &  8.80    \\
        32    & 11.71   &  8.51    \\
        34    & 11.75   &  8.14    \\
        36    & 11.73   &  7.80   \\
        38    & 11.58   &  7.31 \\
        40    & 11.70   &  7.03 \\
        42    & 11.72   &  6.62 \\
        44    & 11.54   &  6.24 \\
        46    & 11.62   &  6.01 \\
        48    & 11.44   &  5.48 \\
        \bottomrule
    \end{tabular}
\end{table}
\begin{figure}[H]
    \centering
    \includegraphics{build/e_einseit.pdf}
    \caption{Messwerte und Ausgleichsgerade für die einseitige Belastung des eckigen Stabs.}
    \label{fig:plot2}
\end{figure}
\noindent In Abbildung \ref{fig:plot2} ist zusätzlich noch eine Ausgleichsgerade, mit einer 
Steigung $m=\SI[per-mode=reciprocal]{0.0764\pm0.001}{\per\meter\squared}$, aufgetragen.
Der Stab besitzt ein Flächenträgheitsmoment von:
\begin{equation*}
  I=\int_Q y^2 dq(y) = \frac{(\SI{0.01}{\meter})^4}{12} = \SI{8.33e-10}{\meter\tothe{4}}
\end{equation*}
Der Stab besitzt ein Elastizitätsmodul von:
\begin{equation*}
  E= \SI{9.334\pm0.122e10}{\newton\per\meter\squared}
\end{equation*}
Die Abweichung errechnet sich hier über Gaußsche Fehlerfortpflanzung.
%
\subsection{Bestimmung des Elastizitätsmoduls mihilfe eines beidseitig aufliegenden Aluminiumtabs}
Im Gegensatz zu den vorausgegangenen Messungen ist die überstehende Länge des Stabs hier nicht $\SI{50}{\centi\meter}$,
sondern $L=\SI{55.5}{\centi\meter}$.
Das Gewicht wurde aufgrund der geringen Auslenkung so groß wie möglich gewählt, hier $m=\SI{4699.8}{\gram}$.
Es ergeben sich folgende, in Tabelle \ref{tab:eckig_beidseit} aufgetragene, Werte:
\begin{table}[H]
    \centering
    \caption{Messwerte für einen beidseitig aufliegenden, eckigen Stab.}
    \label{tab:eckig_beidseit}
    \begin{tabular}{S[table-format=2.0(0)e0] S[table-format=1.2(0)e0] S[table-format=1.2(0)e0] }
        \toprule
        {$x/\si{\centi\meter}$} & {$D_0/\si{\milli\meter}$} & {$D_m/\si{\milli\meter}$} \\
        \midrule
        4       & 9.01  & 8.34  \\
        6       & 9.02  & 8.11  \\
        8       & 9.0   & 7.78  \\
        10      & 8.97  & 7.53  \\
        12      & 9.0   & 7.25  \\
        14      & 8.98  & 7.0   \\
        16      & 8.96  & 6.42  \\
        18      & 8.94  & 6.55  \\
        20      & 8.96  & 6.42  \\
        22      & 8.96  & 6.22  \\
        24      & 8.91  & 6.16  \\
        26      & 8.91  & 6.09  \\
        30      & 8.92  & 6.11  \\
        32      & 8.92  & 6.22  \\
        34      & 8.94  & 6.32  \\
        36      & 8.92  & 6.46  \\
        38      & 8.91  & 6.57  \\
        40      & 8.89  & 6.77  \\
        42      & 8.89  & 7.01  \\
        44      & 8.9   & 7.25  \\
        46      & 8.9   & 7.53  \\
        48      & 8.89  & 7.81  \\
        50      & 6.74  & 5.9   \\
        52      & 6.76  & 6.23  \\
        \bottomrule
    \end{tabular}
\end{table}
\noindent Diese Werte werden in Abbildung \ref{fig:plot3} aufgetragen und gemäß der Vorschrift aus Gleichung \eqref{eq:ausl}
wird ein Curve-Fit erstellt und ebenfalls aufgetragen.
\begin{figure}[H]
    \centering
    \includegraphics{build/e_beidseit.pdf}
    \caption{Messwerte und Ausgleichsgerade für den beidseitig aufliegenden, eckigen Stabs.}
    \label{fig:plot3}
\end{figure}
\noindent Der Fit-Parameter ist hier:
\begin{equation}
    A = \frac{F}{48EI} = \SI{16.71\pm0.13e-3}{\per\meter\squared}
\end{equation}
Das Trägheitsmoment lässt sich gemäß der Gleichung \eqref{eq:fltraegheit} berechnen und ergibt sich zu
\begin{align}
    I &= \int y^2z_0 \symup{dy} \\
      &= \frac{y_0^3z_0}{12},
\end{align}
wobei $y_0=z_0=\SI{1}{\centi\meter}$ die Abmessungen der Querschnittsfläche des Stabs sind.
Der Elastizitätsmodul und seine Unsicherheit lässt sich berechnen mit
\begin{equation}
    E=\frac{mg}{48IA}
\end{equation}
und
\begin{equation}
    \symup{\Delta}E=\frac{mg}{48IA^2}\symup{\Delta}A.
\end{equation}
Es ergibt sich somit ein Wert von
\begin{equation}
    E = \SI{68.90\pm0.54e9}{\newton\per\meter\squared}.
\end{equation}
