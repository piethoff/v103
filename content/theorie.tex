\section{Zielsetzung}
In diesem Versuch soll der Elastizitätsmodul von verschiedenen Metallen bestimmt werden.
\section{Theorie}
\label{sec:Theorie}
Für Festkörper der Länge $L$ kann eine kleine Änderung der Länge $\symup{\Delta}L$
durch einen einwirkenden Druck $\sigma$
mithilfe eines linearen Federkraftgesetzes beschrieben werden:
\begin{equation}
    \sigma = E\frac{\symup{\Delta}L}{L}
\end{equation}
$E$ ist der Elastizitätsmodul und abhängig vom verwendeten Material.
\subsection{Einseitig eingespannter Stab}
Aus diesem Zusammenhang lässt sich weiterhin folgern, wie sich ein Stab verhält, der gebogen wird.
Im ersten, nicht trivialen, Fall ist ein Stab einseitig eingespannt und auf das nicht eingespannte
Ende wirkt eine Kraft und somit auf den gesamten Stab ein Drehmoment.
Dieses hat am Punkt $x$ die Form
\begin{equation}
    M_F = F\cdot(L-x),
\end{equation}
mit $F$ als äußerer Kraft am Punkt $x=L$.
Im Stab selbst entsteht in jeder Querschnittsfläche $Q$ ein gegengerichtetes Drehmoment durch die
Streckung der oberen und Stauchung der unteren Biegungsschichten.
Es folgt
\begin{equation}
    \label{eq:fltraegheit}
    M_\sigma = \int_Q y \sigma(y) \symup{dq},
\end{equation}
mit $y$ als Abstand zur mittleren Schicht auf $Q$.
Mit der Annahme, dass der Stab lokal durch eine Krümmung $R$ beschrieben werden kann,
folgt aus geometrischen Überlegungen eine Variante des Federkraftgesetzes
\begin{equation}
    \sigma(y) = E\frac{y}{R}
\end{equation}
und ein Zusammenhang mit der Auslenkung $D(x)$
\begin{equation}
    \frac{\symup{d}^2D}{\symup{dx}^2} \approx \frac{1}{R}.
\end{equation}
Schlussendlich folgt daraus für die Auslenkung
\begin{equation}
  \label{eq:e_aus}
    D(x) = \frac{F}{2EI}\left(Lx^2-\frac{x^3}{3}\right),
\end{equation}
mit $I$ als Flächenträgheitsmoment:
\begin{equation}
    I = \int_Qy^2\symup{dq}
\end{equation}
\subsection{Beidseitig aufgehängter Stab}
Der Fall, dass der Stab an zwei Punkten aufliegt und die Kraft in der Mitte dieser
Auflagepunkte wirkt, kann auf zwei Biegungen des ersten Typs mit der halben Kraft zurückgeführt werden.
Es ergeben sich folgende Drehmomente für die jeweiligen Stabhälften:
\begin{align}
    M_f =
    \begin{dcases}
        -\frac{F}{2}x       & 0 \leq x \leq \frac{L}{2}\\
        -\frac{F}{2}(L-x)   & \frac{L}{2} \leq x \leq L
    \end{dcases}
\end{align}
Aus den gleichen Annahmen wie oben folgt weiter:
\begin{align}
    \frac{\symup{d}^2D}{\symup{dx}^2} \approx
    \begin{dcases}
        -\frac{F}{2EI} x     & 0 \leq x \leq \frac{L}{2}\\
        -\frac{F}{2EI} (L-x) & \frac{L}{2} \leq x \leq L
    \end{dcases}
\end{align}
Es ergibt sich mit der Forderung nach Stetigkeit und \mbox{$\frac{\symup{d}D}{\symup{dx}}(x=0) = 0$}:
\begin{equation}
    \label{eq:ausl}
    D(x) =
    \begin{dcases}
    \frac{F}{48EI}(3L^2x-4x^3) & 0 \leq x \leq \frac{L}{2}\\
    \frac{F}{48EI}(4x^3-12Lx^2+9L^2x -L^3) & \frac{L}{2} \leq x \leq L
    \end{dcases}
\end{equation}
Somit ist die Auslenkung am Punkt $x$ vollständig beschrieben.
