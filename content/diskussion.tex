\section{Diskussion}
Die Messung beeinhaltet mehrere systematische Fehler.
Einerseits liegt der Taster der Messuhr nicht stabil auf der Oberseite des runden Stabs, weshalb er während der Messung verrutschen kann.
Andererseits ist davon auszugehen, dass die Stäbe durch vorherige Benutzung dauerhauft verformt sind.
Zusätzlich gaben die Messuhren, nachdem sie genullt worden sind, an gleicher Stelle verschiedene Werte aus.
Dies kann auf eine ungenaue Eichung zurückgeführt werden.
Unter einbezug von Literaturwerten ist zu erkennen, dass die gemesssenen Werte die richtige Größenordnung besitzen.
Messing besitzt einen Literaturwert von $\num{78e9}$--$\num{123e9}\si{\newton\per\meter\squared}$\cite{lit_wert}, welchen unsere Messung bestätigt.
Dort ist nachzulesen, dass Aluminium einen Literaturwert von $\SI{70e9}{\newton\per\meter\squared}$\cite{lit_wert} aufweist.
Im Vergleich zeigt sich, dass die Messung unter beidseitiger Belastung näher an diesen Wert liegt.
Somit scheint diese als geeigneter.
Diese Aussage ist bei einer Messung sehr Fragwürdig.
Es müssten mehrere Messungen mit verschiedenen Materialien durchgeführt werden, um dies zu bestägigen.
