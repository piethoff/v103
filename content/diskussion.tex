\section{Diskussion}
\begin{table}[H]
  \caption{Ergebnisse der Messung.}
  \label{tab:mess}
  \centering
  \begin{tabular}{S[table-format=3.2(4)]S[table-format=2.0]S[table-format=2.2(3)]}
    \toprule
    {$E/\SI{e9}{\newton\per\meter\squared}$} & {Literaturwert$/\SI{e9}{\newton\per\meter\squared}$} & {Abweichung$/\si{\percent}$}\\
    \midrule
     111.8\pm0.30 & {78-123}  & {0} \\
     93.34\pm1.22 & 70 & 33.34\pm1.74 \\
     68.90\pm0.54 & 70 & 1.57\pm0.77 \\
    \bottomrule
  \end{tabular}
\end{table}

\noindent Die Messung beinhaltet mehrere systematische Fehlerquellen.
Einerseits liegt der Taster der Messuhr nicht stabil auf der Oberseite des runden Stabs, weshalb er während der Messung verrutschen kann.
Andererseits ist davon auszugehen, dass die Stäbe durch vorherige Benutzung dauerhauft verformt sind.
Zusätzlich gaben die Messuhren, nachdem sie kalibriert worden sind, an gleicher Stelle
verschiedene Werte aus.
Dies kann auf eine ungenaue Eichung zurückgeführt werden.
Unter Einbezug von Literaturwerten ist zu erkennen, dass die gemessenen Werte die richtige
Größenordnung besitzen.
Messing besitzt einen Literaturwert von
\mbox{$\num{78e9}$--$\num{123e9}\si{\newton\per\meter\squared}$\cite{lit_wert}}, welchen unsere
Messung bestätigt. Da als Liiteraturwert eine große Bandbreite an möglichen Werten angegeben ist, weicht der Wert nicht ab.
Weiterhin besitzt Aluminium einen Literaturwert von
$\SI{70e9}{\newton\per\meter\squared}$\cite{lit_wert}.
Im Vergleich zeigt sich, dass die Messung unter beidseitiger Belastung, wie in Tabelle \ref{tab:mess} zu sehen, näher an diesem Wert liegt.
Somit scheint diese Messmethode als geeigneter, jedoch ist diese Aussage ist bei einer Messung
sehr fragwürdig.
Es müssten mehrere Messungen mit verschiedenen Materialien durchgeführt werden, um dies zu bestägigen.
